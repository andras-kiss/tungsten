\documentclass[manuscript=article, journal=jceda8]{achemso}
%\documentclass[journal = jacsat]{achemso}
\usepackage{chemformula} % Formula subscripts using \ch{}
\usepackage[T1]{fontenc} % Use modern font encodings
\usepackage[utf8]{inputenc}
\usepackage[version=3]{mhchem}
\usepackage{gensymb}
\usepackage{upgreek}
\usepackage{pgfplots}
\usepackage{pst-plot}
\usepackage{tikz}
\usepgfplotslibrary{colormaps}
\usetikzlibrary{pgfplots.colormaps}
\usepgfplotslibrary{external} 
\tikzexternalize
\newcommand*\mycommand[1]{\texttt{\emph{#1}}}

\author{János Klucsik}
\affiliation{Department of General and Physical Chemistry, University of Pécs, Ifjúság útja 6, 7622 Pécs, Hungary}

\author{Venkat Rajat Rao Yalamarti}
\affiliation{Department of General and Physical Chemistry, University of Pécs, Ifjúság útja 6, 7622 Pécs, Hungary}

\author{András Kiss}
\affiliation{Department of General and Physical Chemistry, University of Pécs, Ifjúság útja 6, 7622 Pécs, Hungary}
\email{akiss@gamma.ttk.pte.hu}

\title{Demonstration of potentiometric pH determination using the filament of an incandescent lightbulb as indicator electrode}
\keywords{pH, electrode, tungsten, light--bulb, potentiometry}


\begin{document}
\begin{abstract}
One of the earliest concepts of chemistry taught in school is pH. It is usually introduced during the discussion of acids and bases. Students are taught to detect the acidity of substances using pH indicators, usually in the form of indicator paper. While this method is convenient and easy for students to relate to, it does not give any deeper insight into the meaning of pH. To truly comprehend pH, students need to learn to approach the concept from an electroanalytical point of view. The aim of this paper is to avail teachers of an interactive and demonstrative experiment in order to facilitate this transition.

We demonstrate the fabrication of a simple and economical pH indicator electrode using simple household items. The sensing element is made from the filament of an incandescent lightbulb, that is sandwiched between two pieces of Plexiglass using superglue. 

%Several electrodes have been fabricated and used by Chemistry BSc students as part of an electrochemistry special course. The experiment elicited an interest in electroanalytical chemistry among the students, and it helped them to acquire a deeper understanding of pH.

%A simple and robust pH sensitive electrode is presented that can be constructed from cheap and readily available components. The electrode is of the metal/metal--oxide type. The sensing element is the tungsten filament of an incandescent light bulb, whose potential in a solution is dependent on the hydrogen--ion activity. The electrode can be easily built in a high--school chemistry laboratory by students, and can be used to demonstrate the potentiometric determination of pH.
\end{abstract}

\section{Introduction}

pH is one of the most important concepts in chemistry, originally defined by S\o rensen in 1909 \cite{sorensen1909messung} as the negative logarithm of the concentration of hydrogen ion:

\begin{equation}
\textrm{pH} = -\log_{10} c_{\textrm{H}^+}
\end{equation}

At first glance, it looks like a deceptively simple concept. Complications around this definition start to arise when we replace concentration with activity, taking into account that hydrogen ions have charge, therefore their behaviour correlates more closely with activity than concentration:

\begin{equation}
\textrm{pH} = -\log_{10} a_{\textrm{H}^+} = -\log_{10} c_{\textrm{H}^+} \gamma_{H^+}
\end{equation}

where $a_{H^+}$ is the activity and $\gamma_{H^+}$ is the activity coefficient of hydrogen ions. This definition however raises another problem. It includes the activity coefficient of a single ion, which is unmeasurable. It is possible to measure the \emph{mean activity coefficient} only, for example in the case of HCl solution:

\begin{equation}
\gamma_{H^+, Cl^-} = \sqrt{\gamma_{H^+} \gamma_{Cl^-}}
\end{equation}


Even though the Debye--Hückel equation can be used to calculate a theoretical activity coefficient for a single ion, the prevailing view is that if it cannot be measured, then a pH definition that is based on that activity coefficient is an unmeasurable quantity. This logical difficulty was summarized by Bates and Guggenheim in 1960 \cite{bates1960report}. 

Their recommended solution was a convention -- since then it has been known as the \emph{Bates--Guggenheim convention} --, in which the activity of a chosen ionic species is defined using the Debye--Hückel equation, and the rest is calculated using the theoretically obtained value for the single ionic species and the experimentally determined mean activity coefficient. The chosen ion is chloride ion. For instance, the activity coefficient of hydrogen ion can be calculated as:

\begin{equation}
\gamma_{H^+} = \frac{\gamma_{H^+, Cl^-}^2}{\gamma_{Cl^-}}
\end{equation}

This convention has been the basis for the IUPAC (International Union of Pure and Applied Chemistry) recommendation to define pH in 1985 \cite{covington1985definition} and 2002 \cite{buck2002measurement}, which is currently the most up to date definition from IUPAC:

\begin{equation}
\textrm{pH}= -\log_{10}a_{H^+} = -\log_{10}(m_{H^+} \gamma_{m, H^+}/m^\theta)
\end{equation}

where $m_{H^+}$ is the molality and $\gamma_{m, H^+}$ is the molality based activity coefficient of hydrogen ions, $m^\theta$ is the standard molality (1 mol kg$^{-1}$). The activity coefficient of hydrogen ions in the above equation is calculated using the Bates--Guggenheim convention.

The electrochemical cell used in the measurement of the primary pH standards is known as the \emph{Harned Cell} \cite{harned1958activity}.
To measure the pH in such a cell, a conventional procedure was developed at NBS (National Bureau of Standards) \cite{durst1975standardization} and recommended at present by the last IUPAC recommendations \cite{covington2002measurement}.
NIST (National Institute of Standards and Technology) in the U.S. and PTB (Physikalisch-Technische Bundesanstalt) in Germany have presented pH values using the Harned Cell. 

In practice, pH is most commonly measured with a glass electrode.
It has been known for more than a hundred years that the potential difference measured across a glass membrane depends on the ratio of the hydrogen ion activity in the two solutions that the membrane separates \cite{haber1909elektrische, haber1909concerning}.
The glass electrode is usually combined with a reference electrode for convenience.
Such an electrode pair is called the \emph{,,combined glass electrode''}, and this the most common type the students might encounter.
The glass electrode is the best electrochemical sensor and one of the best sensors ever made, with a linear response of over more than 13 orders of magnitude and excellent selectivity.
However, it's not perfect. One of the imperfections is that although to a negligible extent, it responds to cations other than hydrogen ion.
This behaviour is described by the Nikolsky--equation that takes the effect of interfering ions into account with the so-called \emph{,,selectivity coefficient''} \cite{nicolsky1937theory}:

\begin{equation}
E=E^\theta + \frac{RT}{z_iF} \ln \left [ a_i + \sum_{j} \left ( k_{ij}a_j^{z_i/z_j} \right ) \right ]
\end{equation}

There are several other type of electrodes that can be used to measure pH.
One of them is the ionophore based hydrogen ion selective electrode. 
Its function is based on an organic ionophore that selectively complexes hydrogen ions.
When such an ionophore is embedded in a PVC based membrane, the crossmembrane potential difference depends on the activity ratio of hydrogen ion at the two opposing sides of the membrane \cite{goldcamp2010inexpensive}.

There are certain applications where the use of a glass electrode might be challenging or impossible.
These include measuring pH at high temperature or in hydrogen--fluoride solution, or application in the food industry. 
Intensive research has been conducted for several decades to find alternatives to the glass electrode. One such alternative is to use metal/metal-oxide electrodes.
One of the most often used type of these is the Ir/IrO$_2$ electrode \cite{beyenal2004improved}.
The oldest is certainly the Sb/Sb$_2$O$_3$ electrode, its initial characterization dating back to 1923 \cite{uhl1923electrometric}.
It is pH sensitive because hydrogen ions participate in the equilibrium between antimony and its oxide on the surface:

\begin{equation}
        \ce{2Sb_{(s)} + 3H_2O <=> Sb_2O_3 + 6H^+ + 6e^-}
\end{equation}

Another very popular metal/metal-oxide electrode used for pH measurements is the tungsten electrode. Its function is also based on the equilibrium between the metal and its oxide \cite{kriksunov1994tungsten}:

\begin{equation}
        \ce{W_{(s)} + 2H_2O <=> WO_2 + 4H^+ + 4e^-}
\end{equation}

In this paper the construction and usage of a simple and inexpensive tungsten pH sensitive electrode is presented. The electrode can be assembled in a highschool or undergraduate chemistry laboratory by students and it can be used as an aid to demonstrate pH and ion-selective electrodes. The excercise can also be useful as an introduction to the IUPAC definition of pH, since it is closely intertwined with the potentiometric determination of pH.


we must mention j. of chem. ed. papers here about economical and interesting pH electrodes and their use in demonstrating potemtiometric pH determination.




\section{Materials and Methods}

\paragraph{Electrode fabrication.}



To obtain the sensing element of the electrode, a 100 W incandescent lighbulb (Tungsram brand, purchased at local hardware store) was carefully wrapped in cloth and broken with a mallet. The contact wire that still had the filament attached was cut near the stem (Fig. \ref{fig:fabrication}a). The electrode body was prepared from two, 4 mm thick, $\approx$ 0.5 cm $\times$ $\approx$ 2 cm, identically shaped Plexiglass pieces (Fig. \ref{fig:fabrication}b). In one of them, a $\approx$ 1 mm groove was cut longitudinally, in which the filmanet with stem was layed (Fig. \ref{fig:fabrication}c). Then, superglue was applied onto the Plexiglas surface that contained the groove with the filament and stem. The other Plexiglass piece was pressed onto the superglued surface, and was held firmly until the superglue cured (Fig. \ref{fig:fabrication}d). The electrode was sanded and polished from all sides except where the electrical lead, the stem was protruding (Fig. \ref{fig:fabrication}e).

\begin{figure}
\centering
\includegraphics[width=0.9\textwidth]{electrode_fabrication.eps}
\caption{The fabrication process of the tungsten filament electrode. }
\label{fig:fabrication}
\end{figure}

\paragraph{Calibration.}

The fabricated tungsten filament electrode was calibrated in a series of buffers ranging from pH = 3.9 to pH = 9.71. For the first three buffers, with a pH value of 3.9, 4.56 and 5.63, Walpole acetate buffer was used. For the next three, with pH values 6.33, 7.06 and 8.12, Michaelis phosphate buffer was used. For the last three, with pH values of 8.68, 9.37 and 9.71, Michaelis borate buffer was prepared. The pH was measured with a WTW SenTix combined pH electrode connected to a WTW inoLab pH 740p high input impedance pH meter (Xylem Analytics Germany Sales GmbH \& Co. KG, Weilheim Germany). The pH was adjusted by adding small quantities of HCl and NaOH, if it was necessary. The glass and the tungsten filament electrodes were calibrated by measuring their potential against the internal reference electrode of the glass electrode in the buffer series. 


\paragraph{Selectivity study.}

Since the primary ion for the glass and the tungsten filament electrodes is already present in water at an activity that would influence the selectivity study, a buffer had to be used to lower the hydrogen ion activity. Most of the common pH buffers contain alkali metals, and they could not be used to carry the selectivity study, because alkaline and alkaline earth metals are the most common interferences. For this reason, a pH = 8 TRIS buffer was used as a solvent to create the dilution series using KCl, NaCl, CaCl$_2$ and MgCl$_2$. In each cases, the activities ranged from 10$^{-6}$ M to 10$^{-1}$ M with tenfold increments in activity. The electrodes were immersed in each of the solutions and the resulting potential differences were measured and recorded with the already mentioned pH meter.

\section{Hazards}
When salvaging the tungsten filament from the lighbulb, care must be taken. The lightbulb must be wrapped several times in a cloth before it is hit with a mallet. Care must be taken when the filament is picked up from the broken pieces of glass. When the Plexiglass body is prepared, a saw is used to cut the groove, then superglue is used to enclose the filament in the groove. During these steps, close supervision of the students is necessary to avoid any harm. Lastly, HCl and NaOH is used during the preparation of the pH buffer series. If these get in contact with skin or the eye, flush with plenty of water for several minutes.




\section{Results and Discussion}

\begin{figure}[h!]
\centering
\begin{tikzpicture}
\begin{axis}[
legend style={draw=none, at={(0.02,0.98)}, anchor=north west},
	xmin=0.5, xmax=10, width=7cm,height=7cm,xlabel=pCATION, ylabel={E vs. Ag/AgCl/KCl(3 M), mV}, clip marker paths=true, xtick = {0,1,2,3,4,5,6,7,8,9,10}, xminorticks=true, minor x tick num = 1]
\addplot [only marks, domain=0:5.5, mark=*, color=red] table {data/calibration/uveg_h.txt};
\addplot [only marks, domain=0:5.5, mark=*, color=blue] table {data/calibration/uveg_na.txt};
\addplot [only marks, domain=0:5.5, mark=*, color=green] table {data/calibration/uveg_k.txt};
\addplot [only marks, domain=0:5.5, mark=*, color=black] table {data/calibration/uveg_mg.txt};
\addplot [only marks, domain=0:5.5, mark=*, color=purple] table {data/calibration/uveg_ca.txt};
%\addplot [domain=0:3, samples=500, line width=0.1mm] {86.8-x*29.12};
\addlegendentry{H$^+$} 
\addlegendentry{Na$^+$}
\addlegendentry{K$^+$}
\addlegendentry{Mg$^{2+}$}
\addlegendentry{Ca$^{2+}$}
\node[anchor=north east] at (rel axis cs:0.98,0.98) {A};
\node[anchor=south west] at (rel axis cs:0,0) {$\Delta$ S = 54.788 mV / pH};
\node[anchor=south west] at (rel axis cs:0,0.1) {R$^2$ = 0.9998};
\end{axis}
\end{tikzpicture}
\begin{tikzpicture}
\begin{axis}[
	legend style={draw=none, at={(0.02,0.98)}, anchor=north west},
	xmin=0.5, xmax=10,
	width=7cm,height=7cm,
	xlabel=pCATION, ylabel={E vs. Ag/AgCl/KCl(3 M), mV},
	clip marker paths=true,
	xtick = {0,1,2,3,4,5,6,7,8,9,10}, xminorticks=true, minor x tick num = 1]
\addplot [only marks, domain=0:5.5, mark=*, color=red] table {data/calibration/izzo_h.txt};
\addplot [only marks, domain=0:5.5, mark=*, color=blue] table {data/calibration/izzo_na.txt};
\addplot [only marks, domain=0:5.5, mark=*, color=green] table {data/calibration/izzo_k.txt};
\addplot [only marks, domain=0:5.5, mark=*, color=black] table {data/calibration/izzo_mg.txt};
\addplot [only marks, domain=0:5.5, mark=*, color=purple] table {data/calibration/izzo_ca.txt};
%\addplot [domain=0:3, samples=500, line width=0.1mm] {86.8-x*29.12};
\addlegendentry{H$^+$} 
\addlegendentry{Na$^+$}
\addlegendentry{K$^+$}
\addlegendentry{Mg$^{2+}$}
\addlegendentry{Ca$^{2+}$}
\node[anchor=north east] at (rel axis cs:0.98,0.98) {B};
\node[anchor=south west] at (rel axis cs:0,0) {$\Delta$ S = 53.207 mV / pH};
\node[anchor=south west] at (rel axis cs:0,0.1) {R$^2$ = 0.9817};
\end{axis}
\end{tikzpicture}
\caption{Calibration and selectivity study of the glass electrode (A), and the lightbulb electrode (B).}
\label{fig:calibration}
\end{figure}


\section{Conclusion}
	
The construction of an economical pH sensitive electrode was presented that is highly selective towards hydrogen ions. The potentiometric cell utilizing the electrode can be used as an engaging demonstration of the potentiometric determintation of pH. The experiment has been carried out by 1st year chemistry BSc students with electrodes that they have prepared themselves.



\begin{acknowledgement}
The project has been supported by the European Union, co-financed by the European Social Fund Grant no.: EFOP-3.6.1.-16-2016-00004 entitled by Comprehensive Development for Implementing Smart Specialization Strategies at the University of Pécs. The work was supported by the Hungarian Research Grant: NKFI No.: K125244.
\end{acknowledgement}

\bibliography{tungsten}
\end{document}
