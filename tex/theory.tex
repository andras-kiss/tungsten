Since the glass electrode cannot be effectively miniaturized due to the reasons detailed in the previous sections, intensive research has been conducted for several decades to improve metal/metal-oxide electrodes.
One of the most often used type of these is the Ir/IrO$_2$ electrode \cite{beyenal2004improved}.
The oldest is certainly the Sb/Sb$_2$O$_3$ electrode, its initial characterization dating back to 1923 \cite{uhl1923electrometric}.
It is based on the equilibrium between antimony and the antimony-oxide on its surface.
It is pH sensitive because hydrogen ions participate in the equilibrium:

\begin{equation}
        \ce{2Sb_{(s)} + 3H_2O <=> Sb_2O_3 + 6H^+ + 6e^-}
\end{equation}

\begin{equation}
        \ce{Sb_{(s)} + 3OH^- <=> Sb(OH)_3 + 6e^-}
\end{equation}

Using the Nernst-equation to describe the relationship between [Sb$^{3+}$] and the measured electrode potential \cite{kurzweil2009metal}:

\begin{equation}
E = E^\theta - \frac{RT}{3F}\ln a_{\textrm{Sb}^{3+}}
\end{equation}

Which can be rewritten by using $K_W$, the water ion product, and $K_L$, the solubility product of antimony hydroxide as \cite{kurzweil2009metal}:

\begin{equation}
E = E^\theta - \frac{RT}{3F}\ln \frac{K_L}{K_W} + \frac{RT}{F}\ln a_{\textrm{H}^+}
= E^{\theta '} - \frac{RT}{0.4343F} \textrm{pH}  
\end{equation}
because [Sb$^{3+}$]=[H$^+$]$\cdot K_L / K_W$.

The main reason this particular electrode is so popular is that the melting point of antimony and the softening point of borosilicate glass are similar ($T_{m, \textrm{Sb}} = 630.63~\celsius$, $T_{m, glass}\approx 700~\celsius$), and manufacturing them is relatively easy with standard glass blowing techniques.

Another very popular metal/metal-oxide electrode used for pH measurements is the tungsten electrode. Its function is also based on the equilibrium between the metal and its oxide:
\begin{equation}
        \ce{W_{(s)} + 3H_2O <=> WO_3 + 6H^+ + 6e^-}
\end{equation}
